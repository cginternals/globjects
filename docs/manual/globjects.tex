\documentclass{article}

\usepackage[pdfborder={0 0 0}]{hyperref}
\usepackage{listings}

\begin{document}

\title{globjects: An OpenGL Objects Wrapper}
\author{Willy Scheibel, Daniel Limberger, Stefan Buschmann\\CG Internals}

\maketitle

\begin{abstract}
\noindent
The \emph{globjects} library provides object-oriented interfaces to the OpenGL API version 3.0 and higher. Reoccuring tasks are automated and features the current driver doesn't provide are simulated through earlier interfaces or even emulated with own implementations. One goal is a leaky abstraction of OpenGL, i.e., globjects can be used in conjunction with plain OpenGL API calls, e.g., from \emph{glbinding}.

\end{abstract}

\setcounter{tocdepth}{2}
\tableofcontents

\newpage

\section{Overview}

globjects is a library providing object-oriented interfaces to the OpenGL API. Targeted OpenGL features (versions in the language of the OpenGL API) are 3.0 and above.
The main goals are much reduced code to use OpenGL in your rendering software and fewer errors due to the underlying glbinding and abstraction levels on top. Typical processes are automated and missing features in the used OpenGL driver are partially simulated or even emulated.

\subsection{globjects to plain OpenGL API Usage Comparison}

globjects should reduce the amount of code required in a rendering system. An example of a typical part of an OpenGL program is as follows:

\begin{lstlisting}[language=c++,frame=single,basicstyle=\footnotesize]
GLuint program = glCreateProgram();
GLuint vertexShader = glCreateShader(GL_VERTEX_SHADER);
GLuint fragmentShader = glCreateShader(GL_FRAGMENT_SHADER);

glShaderSource(vertexShader, 1, &vertexShaderSource, nullptr);
glShaderSource(fragmentShader, 1, &fragmentShaderSource, nullptr);

glCompileShader(vertexShader);
glCompileShader(fragmentShader);

glAttachShader(program, vertexShader);
glAttachShader(program, fragmentShader);

glLinkProgram(program);

glUseProgram(program);
glUniform2f(glGetUniformLocation(program, "extent"), 1.0, 0.5);
\end{lstlisting}

\noindent The same functionality and optional error checks in terms of \lstinline|glGetError| and compiler and linker status can be achieved using globjects with the following code:
\begin{lstlisting}[language=c++,frame=single,basicstyle=\footnotesize]
Program* program = new Program();

program->attach(
  Shader::fromString(GL_VERTEX_SHADER, vertexShaderSource), 
  Shader::fromString(GL_FRAGMENT_SHADER, fragmentShaderSource)
);

program->setUniform<glm::vec2>("extent", glm::vec2(1.0, 0.5)));
\end{lstlisting}

\subsection{Philosophy}

One main goal of globjects is to be as strict as possible. This means, that everything that isn't generalizable is not integrated. It provides the current interface the OpenGL API suggests and provides fallback implementations for drivers that cannot provide the newest API. Further, globjects is designed for leaky abstraction, i.e., direct OpenGL API calls or other wrapping libraries can be used together with globjects. For dynamically allocated memory, globjects rely on reference counting through smart pointers. To allow for multiple contexts and even shared contexts, globjects uses registries to manage different behavior.

\subsection{Scope}

The scope of globjects are the features currently supported in the newest OpenGL API version and some features that are currently extensions but are likely to find their way to the OpenGL Core. Features that were used before OpenGL 3.0 are explicitly not covered. Likewise, features that are only supported by single vendors are likely to get omitted but are subject for discussion.

An example list of features included:
\begin{itemize}
	\item State / Capability (e.g., Point Size, Depth Test)
	\item Immediate Error Checking
	\item Uniform
	\item Renderbuffer
	\item Framebuffer
	\item Texture (including Bindless and Storage)
	\item Image
	\item Sampler
	\item Shader
	\item Program (including Compute Shaders)
	\item Query
	\item Transform Feedback
	\item Vertex Buffer / Index Buffer (and uniform buffer, atomic counter buffer, shader storage buffer)
	\item Vertex Array
	\item Generic Attributes and Bindings
	\item Shading Language Includes
	\item Program Binaries
\end{itemize}

\noindent An example list of features excluded:
\begin{itemize}
	\item Display Lists
	\item Immediate Mode
	\item Client Vertex Buffer
\end{itemize}

\noindent An example list for subjects to discussion:
\begin{itemize}
	\item Command Lists
	\item Bindless Buffer and storage
\end{itemize}

\subsection{Similar Projects}

The idea to wrap the OpenGL API using a language that supports object-oriented semantics is far from new. Nevertheless, globjects differs notable from competitors.
As globjects don't want to compete with complete rendering engines (Unity, Unreal Engine, CryEngine, FrostBite, Oger, OpenSceneGraph, Magnum, \dots), the list of similar projects is limited to sole object-oriented wrappers.

\paragraph{OGLplus}

\url{http://oglplus.org/}

\paragraph{glcxx}

\url{https://holtrop.github.io/glcxx/doxygen/annotated.html}

\paragraph{OOGL}

\url{https://github.com/Overv/OOGL}

\paragraph{gloo}

\url{http://vispy.org/gloo.html}

\paragraph{JOGL}

\url{http://jogamp.org/jogl/www/}

\paragraph{OpenFrameworks gl}

\url{http://openframeworks.cc/documentation/gl/}

\paragraph{ClanLib GL}

\url{https://github.com/sphair/ClanLib}

\paragraph{glpp}

\url{https://github.com/sque/glpp}

\paragraph{GLT}

\url{http://www.nigels.com/glt/}

\paragraph{GLWrappers}

\url{http://idav.ucdavis.edu/~okreylos/ResDev/Vrui/Documentation/OverviewGLWrappers.html}

\paragraph{Qt5 OpenGL}

\url{http://doc.qt.io/qt-5/qtopengl-index.html}

\section{Project Setup}

The structure of globjects is driven by the design decisions in means of compile-time and run-time dependencies as well as the intended development and deployment use-cases. The project is permanently hosted on \url{https://github.com/cginternals/globjects}.

\subsection{Required Build Environment}

To build and use globjects on your system, a C++11 compatible compiler is required. As a few newer C++11 features are used, not every C++11 compiler is suitable. The list and versions of supported and tested compilers is:
\begin{itemize}
	\item MSVC $\geq$ 2013 Update 3
	\item GCC  $\geq$ 4.8
	\item clang $\geq$ 3.3
	\item AppleClang $\geq$ 6
\end{itemize}
Further, your system should also have the following pieces of software:
\begin{itemize}
	\item CMake $\geq$ 3.0
	\item git $\geq$ 1.8
	\item An OpenGL driver with OpenGL $\geq$ 3.0
	\begin{itemize}
		\item Nvidia
		\item AMD
		\item Intel
		\item Mesa
	\end{itemize}
\end{itemize}

\subsection{Project Dependencies}

Compile-time dependencies of globjects are:
\begin{itemize}
	\item glbinding (required, \url{https://github.com/cginternals/glbinding})
	\item cpplocate (optional, \url{https://github.com/cginternals/cpplocate})
	\item GLFW (optional, \url{www.glfw.org/})
	\item Qt5 (optional, \url{http://www.qt.io/})
\end{itemize}

Run-time dependencies of globjects are:
\begin{itemize}
	\item glbinding (required)
	\item GLFW (optional)
	\item Qt5 (optional)
	\item An OpenGL driver (required)
\end{itemize}

\noindent All optional dependencies are used within the examples. If you don't want to compile or run the examples than you don't need anything besides glbinding and an OpenGL driver.

\subsection{File Structure}

The globjects project is using git as code repository and GitHub as project management system. The file structure is based on cmake-init\footnote{cmake-init project page: \url{https://github.com/cginternals/cmake-init}}. This results in the following root directory structure:
\begin{description}
	\item[cmake] Additional find scripts and functions used by the cmake project setup.
	\item[codegeneration] Fallback files if the cmake version cannot generate certain files.
	\item[data] Run-time data for examples.
	\item[deploy] Cmake-scripts for deployment.
	\item[docs] Documentation, including doxygen and this one.
	\item[source] Actual source code of the \emph{globjects} library, \emph{tests} and \emph{examples}.
\end{description}

\section{Concepts}

globjects provides classes for every main concept of the current OpenGL API. For every concept the according globjects class provides the newest interface and emulates or simulates the behavior if the newest interface of OpenGL is not available in your driver.

\subsection{State}

A \lstinline|State| object represents the accumulation of all global states an OpenGL context possesses. Mainly, this means a collection of \lstinline|StateSettings| (states with non-boolean values) and a collection of \lstinline|Capabilities| (boolean states).

\subsubsection{Capability}

A \lstinline|Capability| object represents a boolean state in the global OpenGL state machine.

\subsubsection{StateSetting}

A \lstinline|StateSetting| object represents compound global states that consist of one or more parameters that are set through a specific function call (e.g., \lstinline|glPointSize|).

\subsection{DebugMessage}

A \lstinline|DebugMessage| object represents an OpenGL message with a type, severity and message string. The interface can also manage the generation and propagation of messages throughout the OpenGL API.

\subsection{Error}

A \lstinline|Error| object represents an error state of the state machine.

\subsection{Uniform}

A \lstinline|Uniform| object represents a uniform value of a program. All uniforms are automatically re-uploaded after a program relink.

\subsection{Renderbuffer}

A \lstinline|Renderbuffer| object represents a renderbuffer that is attachable to framebuffer objects.

\subsection{Framebuffer}

A \lstinline|Framebuffer| object represents a framebuffer objects with attachable renderbuffers and textures.

\subsubsection{Framebuffer Attachment}

A \lstinline|FramebufferAttachment| object represents the common superclass between renderbuffer and texture attachments.

\subsection{Texture}

A \lstinline|Texture| object represents 

\subsubsection{TextureHandle}

A \lstinline|TextureHandle| object represents 

\subsection{Image}

A \lstinline|Image| object represents 

\subsection{Shader}

A \lstinline|Shader| object represents 

\subsection{Program}

A \lstinline|Program| object represents 

\subsubsection{ProgramBinary}

A \lstinline|ProgramBinary| object represents 

\subsection{Query}

A \lstinline|Query| object represents 

\subsection{TransformFeedback}

A \lstinline|TransformFeedback| object represents 

\subsection{Buffer}

A \lstinline|Buffer| object represents 

\subsection{Uniform}

A \lstinline|Uniform| object represents 

\subsection{VertexArray}

A \lstinline|VertexArray| object represents 

\subsection{VertexAttributeBinding}

A \lstinline|VertexAttributeBinding| object represents 

\section{Strategies}

globjects uses the design pattern \emph{strategy} to support multiple implementations for one interface using different OpenGL functions and backends. They have to be differentiated for each use case as they are developed asynchronously and every driver can decide which API extension to include.

\subsection{Buffer}

The \lstinline|Buffer| strategies handle

\subsection{Debug}

The \lstinline|Debug| strategies handle

\subsection{Framebuffer}

The \lstinline|Framebuffer| strategies handle

\subsection{ObjectName}

The \lstinline|ObjectName| strategies handle

\subsection{ProgramBinary}

The \lstinline|ProgramBinary| strategies handle

\subsection{ShadingLanguageInclude}

The \lstinline|ShadingLanguageInclude| strategies handle

\subsection{Texture}

The \lstinline|Texture| strategies handle

\subsection{Uniform}

The \lstinline|Uniform| strategies handle

\subsection{VertexAttributeBinding}

The \lstinline|VertexAttributeBinding| strategies handle

\section{Automated Processes}

As one main goal of globjects is the usage in modern early-prototyping applications it supports some generalizable automation concepts.

\subsection{Objects Bindings}



\subsection{Error Checking}



\subsection{Shader Compilation}

 

\subsection{Program Linkage}



\subsection{Uniform Updates}



\section{Multi-Threading and Multi-Context Support}


\section{Remarks}

\subsection{Limitations}

\subsection{Future Development}



\end{document}